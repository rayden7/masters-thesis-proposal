% !TEX root = ../dkilleffer-thesis-proposal.tex
%
\chapter{Design}
\label{sec:design}


\section{Overview}
\label{sec:design:overview}

The prototype will take the form of a website, and will be created using \hyperref[glossary:node.js]{Node.js}, \hyperref[glossary:express.js]{Express.js}, HTML, CSS, and JavaScript for the backend.  The frontend will make use of \hyperref[glossary:bootstrap]{Bootstrap} for styling and responsive layout, as well as \hyperref[glossary:react.js]{React.js} for dynamic user interface management, particularly for managing video annotations.  Additional libraries may be used to aid with the video player display and management.

\section{Technologies Used}
\label{sec:overview:technologies-used}

In contrast to some earlier work, a key goal of this prototype is for the tool to be a purely online system, allowing for web-based collaboration between users.  Another core requirement of the system is that a faceted search be supported to allow for easy discovery of videos of interest to users.  To support the faceted searching of videos, \hyperref[glossary:elasticsearch]{Elasticsearch} will be used as a document repository of all video annotation data.  When users add annotations to a video, that metadata will be added to Elasticsearch and made available for searches on videos, as well as be used to display top categories of videos.

