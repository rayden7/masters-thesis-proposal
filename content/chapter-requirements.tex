% !TEX root = ../dkilleffer-thesis-proposal.tex
%
\chapter{Requirements}
\label{sec:requirements}

\section{Overview}
\label{sec:requirements:overview}

The Social Video platform prototype will allow for users to upload videos, define groups of individuals that are invited to annotate that video, create actual annotations on videos, enable administrators to "approve" annotations prior to those annotations being made public and searchable, and provide a rich, faceted search interface to find and locate videos of interest.

\section{Details}
\label{sec:requirements:details}

Detailed requirements for the prototype are as follows:

%\begin{itemize}[noitemsep]
\begin{itemize}[leftmargin=*]
\item \textbf{User Authentication and Authorization} \\
	New users may create an account and request a particular level of access (\textbf{Regular User} or \textbf{Administrative User}).  \textbf{Regular Users} are allowed to annotate videos, but \textbf{Administrative Users} can upload videos, create annotations, and "approve" annotations that \textbf{Regular Users} have added prior to those being made public and searchable.
\item \textbf{User-Defined Groups} \\

\item \textbf{Rich Annotations} \\

\item \textbf{Video Upload} \\
	
\item \textbf{Faceted Search} \\

\item \textbf{Video Playlists} \\

\end{itemize}



\section{Technologies Used}
\label{sec:requirements:technologies-used}

In contrast to some earlier work, a key goal of this prototype is for the tool to be a purely online system, allowing for web-based collaboration between users.  



%\cleanchapterquote{Innovation distinguishes between a leader and a follower.}{Steve Jobs}{(CEO Apple Inc.)}
%
%\Blindtext[2][1]
%
%\section{Requirements Section 1}
%\label{sec:requirements:sec1}
%
%\Blindtext[1][2]
%
%\begin{figure}[htb]
%	\includegraphics[width=\textwidth]{gfx/Clean-Thesis-Figure}
%	\caption{Figure example: \textit{(a)} example part one, \textit{(c)} example part two; \textit{(c)} example part three}
%	\label{fig:requirements:example1}
%\end{figure}
%
%\Blindtext[1][2]

%\section{Requirements Section 2}
%\label{sec:requirements:sec2}
%
%\Blindtext[1][2]

%\begin{figure}[htb]
%	\includegraphics[width=\textwidth]{gfx/Clean-Thesis-Figure}
%	\caption{Another Figure example: \textit{(a)} example part one, \textit{(c)} example part two; \textit{(c)} example part three}
%	\label{fig:requirements:example2}
%\end{figure}
%
%\Blindtext[2][2]

%\section{Requirements Section 3}
%\label{sec:requirements:sec3}
%
%\Blindtext[4][2]

%\section{Conclusion}
%\label{sec:requirements:conclusion}
%
%\Blindtext[2][1]
